\documentclass[a4paper, 14pt]{extarticle}
\usepackage[utf8]{inputenc}
\usepackage[T2A]{fontenc}
\usepackage[russian]{babel}
\usepackage{epigraph}
\usepackage{fourier}
\usepackage{array}
\usepackage{makecell}
\usepackage{setspace,amsmath}
\usepackage{multirow}
\usepackage{graphicx}
\usepackage{indentfirst}
\usepackage{tocloft}
\renewcommand{\cftsecleader}{\cftdotfill{\cftdotsep}}
\usepackage[linktocpage=true]{hyperref}
\usepackage{color}
\usepackage{listings}
%\usepackage{indentfirst}
\usepackage[normalem]{ulem}
\useunder{\uline}{\ul}{}
\usepackage[left=20mm, top=15mm, right=15mm, bottom=15mm, nohead, footskip=10mm]{geometry}

\renewcommand\theadalign{bc}
\renewcommand\theadfont{\bfseries}
\renewcommand\theadgape{\Gape[4pt]}

\graphicspath { {./} }

\hypersetup{
    colorlinks=true,
    linkcolor=blue,
    filecolor=magenta,      
    urlcolor=cyan,
}

\begin{document}
\begin{titlepage}
\begin{center}
\large{Московский государственный университет им. М.В. Ломоносова}\\
\vspace{0.5cm}
\normalsize{Факультет вычислительной математики и кибернетики}\\
\vspace{0.25cm}
\normalsize{Кафедра автоматизации систем вычислительных комплексов}\\
\vspace{1.5cm}
\begin{figure}[h]
	\centering
	\includegraphics[width=0.25\textwidth]{logo}
\end{figure}
\vspace{3cm}
\huge{Анализ производительности канала CSMA}
\end{center}
\begin{flushright}
\vspace{4cm}
Фролов Александр\\
321 группа
\end{flushright}
\vspace{4cm}
\begin{center}
Москва\\
2021\\
\end{center}
\end{titlepage}
\thispagestyle{empty}
 
\newpage
\tableofcontents
\newpage

\newpage
\section{Постановка задачи}
Имеется сеть, в которой $N$ хостов подключено к каналу с общим доступом (CSMA).
Длина канала равна $30$ метров.
Скорость передачи $100$ Мбит/с.
На каждом хосте работает одно и то же приложение, в бесконечном цикле отправляющее пакет длиной $1500$ байт (не дожидаясь отправки предыдущего пакета!).

Интервал времени между пакетами – случайная величина с экспоненциальным распределением и средним значением ($\frac{1}{\alpha}$).
Интервал моделирования $10$ с.
Каждый сетевой адаптер имеет буфер для неотправленных пакетов.
Предполагается, что коллизий нет, но занятость канала проверяется адаптером.

Если канал занят, то момент повтора передачи определяется по правилам стандарта IEEE 802.3 для случая коллизии.
Найти
\begin{enumerate}
    \item среднюю долю потерянных пакетов.
    \item среднее и максимальное число пакетов в буфере адаптера.
\end{enumerate}
Найти наибольшее $N$, при котором не будет потерь пакетов для $\alpha = 10, 50, 100$
Какова максимальная длина очереди для этих $N$?
Будет ли очередь расти неограниченно?

\section{Запуск программы}
\begin{itemize}
    \item Установить ns3.
    \item Поместить файл программы в директорию ns-allinone-3.35/ns-3.35/scratch
    \item Пыполнить комманду ns-allinone-3.35/ns-3.35/waf --run scratch/task4 -v
\end{itemize}

\section{Конфигурация вычислительной системы}
\begin{itemize}
    \item Процессор Intel(R) Core(TM) i3-6006U CPU @ 2.00GHz с 4 ядрами;
    \item 12 ГБ оперативной памяти;
    \item Операционная система Ubuntu x64.
\end{itemize}

\section{Математическая модель}
Размер пакета: $1500$ Байт = $12000$ бит

Скорость передачи: $100$ МБит/с = $2^20 * 10^2$ бит/с

Задержка пакетизации: $\frac{12000}{10^2 * 2^20} = 0,000114$ с

Скорость распространения в линии: $3 * 10^7$ м/с

Задержка распространения: $\frac{30}{3 * 10^7}$ = $10^{-6}$ с

Общая задержка: $0,000145$ с

Средний интервал отправки пакетов: $\frac{1}{\alpha}$

Таким образом, за $10$ секунд канал может обработать $\frac{10}{0,000145} = 68966$ пакетов.
Чтобы гарантированно забить очередь, нужно решить следующее неравенство:
\[68966 + N * b \leq N * \frac{10}{\frac{1}{\alpha}} = 10\alpha N\]
Откуда:
\[N \geq \frac{68966}{10\alpha - b},\]
где $b$ - размер буфера.

Исходя из этого возьмем размер буфера равным $10$ и получим:
\begin{align*}
    \alpha = 10  \implies N = 767 \\
    \alpha = 50  \implies N = 141 \\
    \alpha = 100 \implies N = 70  \\
\end{align*}

Выведем формулу для бесконечно растущей очереди при заданном $N$:
\[68966 + N * b \leq 10\alpha N \implies b \leq T\left(\alpha - \frac{1}{0,000145N}\right)\]
Таким образом, если $\alpha \leq \frac{1}{0.000145N}$, то очередь не будет расти. Иначе, очередь будет расти бесконечно.
Оценим $N$:
\[\alpha \leq \frac{1}{0,000145N} \implies N \leq \frac{1}{0,000145\alpha}\]
Отсюда, для заданных $\alpha$ имеем:
\begin{align*}
\alpha = 10 \implies N = 690 \\
\alpha = 50 \implies N = 138 \\
\alpha = 100 \implies N = 69 \\
\end{align*}
\section{Результаты выполнения программы}
\begin{center}
    \begin{tabular}{ c | c | c | c | c | c | c }
    $\alpha$ & N & \thead{Пакетов \\ отправленно} & \thead{Пакетов \\ сброшено} & \thead{Доля потерь} & \thead{Средний \\ размер \\ очереди} & \thead{Максимальный \\ размер \\ очереди} \\
    \hline
     10 & 800 &  81202 &  5463 &   0,0673 & 3,43021 & 10 \\ 
     10 & 700 &  71041 &    0  &   0,0000 & 1,08435 &  7 \\ 
     10 & 750 &  76127 &   22  &   0,0003 & 1,36773 & 10 \\ 
     10 & 725 &  73607 &    2  &  2,7e-05 & 1,14853 & 10 \\ 
     10 & 713 &  72400 &    0  &      0.0 & 1,11010 &  9 \\
     10 & 719 &  73017 &    0  &      0.0 & 1,12512 & 10 \\
     10 & 722 &  73311 &    0  &      0.0 & 1,13875 & 10 \\
     10 & 724 &  73514 &    0  &      0.0 & 1,14276 & 10 \\
    \hline
     50 & 200 & 103643 & 58557 & 0,564988 & 6,68366 & 10 \\ 
     50 & 100 &  51871 &     3 &  5,8e-05 & 1,18203 & 10 \\ 
     50 & 90  &  46696 &     0 &      0,0 & 1,15560 &  9 \\ 
     50 & 95  &  49220 &     0 &      0,0 & 1,16900 &  9 \\ 
     50 & 99  &  51356 &     0 &      0,0 & 1,18072 &  9 \\ 
    \hline
    100 & 100 & 105978 & 64475 & 0,608381 & 6,52413 & 10 \\ 
    100 &  50 &  53103 &    13 &   0,0002 & 1,39619 & 10 \\ 
    100 &  45 &  47949 &     4 &  8,3e-05 & 1,35018 & 10 \\ 
    100 &  44 &  46906 &     2 &  4,3e-05 & 1,33800 & 10 \\ 
    100 &  43 &  45793 &     0 &      0.0 & 1,32308 & 10 \\ 
    \end{tabular}
\end{center}

$N$ практическое:
\begin{align*}
&\alpha =  10 => N = 724 => \text{ очередь будет расти бесконечно} \\
&\alpha =  50 => N = 100 => \text{ очередь не будет расти} \\
&\alpha = 100 => N =  43 => \text{ очередь не будет расти}
\end{align*}

Таким образом, максимальный размер очереди равен $10$.

\section{Вывод}
В ходе выполнения задания была построена модель, было проведено моделирование при помощи библиотеки ns3, а также проведены измерения средней и максимальной загрузки очереди хостов в модели CSMA, доли потерь пакетов при максимальной загрузке канала связи.

Так же было получено, что практическое $N$ меньше чем теоретическое $N$.
Это связано с тем, что задержки представляют собой случайные числа, а значит, в канале будут возникать перегрузки.

\end{document}
